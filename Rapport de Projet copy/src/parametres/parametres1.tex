% Paramètres globaux
    \definecolor{linkborder_Color}{rgb}{0.1329,0.247,0.467}
    \definecolor{urlborder_Color}{rgb}{0,1,1}
    \definecolor{url_Color}{rgb}{1,0,0}

    \hypersetup
    {
        %author = {Thomas AUBIN},
        %pdftitle = {Nom du Projet - Sujet du Projet},
        pdfsubject = {Mémoire de Projet},
        pdfkeywords = {Tag1, Tag2, Tag3, \dots},
        pdfstartview={FitH},
        linkbordercolor = {linkborder_Color}
        %urlbordercolor = \color{urlborder_Color},
        %urlcolor = \color{url_Color}
    }

% Format des niveaux
    % Chiffres romains pour les chapitres
        \renewcommand{\thechapter}{\Roman{chapter}}
        \setlength{\cftchapnumwidth}{\widthof{\large\bfseries{}XVIII}}
        \setlength{\cftsecnumwidth}{\widthof{\large\bfseries{}XVIII}}
        \setlength{\cftsubsecnumwidth}{\widthof{\large\bfseries{}XVIIIII}}
        \setlength{\cftsubsubsecnumwidth}{\widthof{\large\bfseries{}XVIIIIII}}
        \setlength{\cfttabnumwidth}{\widthof{\large\bfseries{}XVIIII}}
        \setlength{\cftfignumwidth}{\widthof{\large\bfseries{}XVIIII}}
    % Rajouter les numérotation pour les \paragraphe et \subparagraphe
        \setcounter{secnumdepth}{4}
        \setcounter{tocdepth}{4}


% Format de l'environnement abstract
    \renewcommand{\abstractnamefont}{\normalfont\Large\bfseries}
% Définir une liste d'équations
    \newcommand{\listequationsname}{Table des Équations}
    \newlistof{myequations}{equ}{\listequationsname}
    \newcommand{\myequations}[1]
    {%
        \addcontentsline{equ}{myequations}{\protect\numberline{\theequation}\hspace{3 mm} #1}\par
    }

% Format des tableaux
    \newcolumntype{u}{>{\columncolor{PaleTurquoise1}} p{0.0727\textwidth}}
    \newcolumntype{d}{>{\columncolor{LightBlue1}} p{0.0727\textwidth}}
    \newcolumntype{t}{>{\columncolor{Khaki1}} p{0.0677\textwidth}}
    \newcolumntype{q}{>{\columncolor{Khaki2}} p{0.0727\textwidth}}
    \newcolumntype{U}{>{\columncolor{PaleTurquoise1}} p{0.13\textwidth}}
    \newcolumntype{D}{>{\columncolor{LightBlue1}} p{0.13\textwidth}}
    \newcolumntype{T}{>{\columncolor{Khaki1}} p{0.13\textwidth}}
    \newcolumntype{Q}{>{\columncolor{Khaki2}} p{0.13\textwidth}}

% Format des codes
    \renewcommand{\lstlistlistingname}{Table des éléments de code}

    \definecolor{codegreen}{rgb}{0,0.6,0}
    \definecolor{codegray}{rgb}{0.5,0.5,0.5}
    \definecolor{codepurple}{rgb}{0.58,0,0.82}
    \definecolor{backcolour}{rgb}{0.95,0.95,0.92}

    \lstdefinestyle{mystyle}{
        backgroundcolor=\color{backcolour},
        frame = single,
        commentstyle=\color{codegreen},
        keywordstyle=\color{magenta},
        numberstyle=\tiny\color{codegray},
        stringstyle=\color{codepurple},
        basicstyle=\ttfamily\footnotesize,
        breakatwhitespace=false,
        breaklines=true,
        captionpos=b,
        keepspaces=true,
        numbers=left,
        numbersep=5pt,
        showspaces=false,
        showstringspaces=false,
        showtabs=false,
        tabsize=2,
        language={Python}
    }
    \lstset{style = mystyle}
% Définir une ligne horizontale
    \newcommand{\HRule}{\rule{\linewidth}{0.5mm}}
% Bibliographie
    \addbibresource{main.bib}