\renewcommand{\abstractname}{Conclusion}
\addcontentsline{toc}{chapter}{Conclusion}
\begin{abstract}
    La compétition \textit{Snake Strikes Back} constitue une entrée en matière intéressante pour
    des débutants en Machine Learning. Le problème posé est simple : les données sont
    normalisées, complètes, exclusivement numériques, et la classification est binaire.
    Pourtant, elle révèle que plusieurs approches sont possibles et qu'aucune ne permet de
    mettre complètement en échec le DoppelGanger.

    L'équipe a connu un lancement difficile en mettant plusieurs semaines à télécharger les
    données nécessaires à la compétition. Ce retard pris au début du projet s'est inscrit en
    parallèle d'une découverte totale du Machine Learning et de difficultés de coordination sur les
    différents aspects du projet. De ce fait, de nombreuses hypothèses restent en suspens. En
    particulier, il est regrettable de ne pas avoir concrétisé les propositions exprimées dans
    les parties \ref{nettoyage}, \ref{regression} et \ref{casparT3}. Une autre explication aux
    scores obtenus est la mauvaise utilisation des métriques d'évaluation des classifieurs. En
    effet celles-ci auraient mérité un croisement plus important pour étudier plus en profondeur
    la forte présence de faux positifs.

    Enfin, sous l'hypothèse d'un lancement moins tardif et d'une montée en compétences plus
    rapide, l'équipe aurait probablement obtenu de meilleurs résultats en considérant des
    approches par Deep Learning. Il est notamment dommage de s'être restreint à une approche
    black-box alors que la compétition permettait une approche white-box, avec un accès complet
    aux scripts du DoppelGanger. Par ailleurs, l'utilisation de réseaux de neurones aurait pu
    être envisagée pour définir plus finement les Shadow Models, et définir de nouveaux modèles
    de classification. Cette approche aurait nécessité un approfondissement de la bibliographie,
    principalement pour l'article \cite{doppelGANger}.
\end{abstract}