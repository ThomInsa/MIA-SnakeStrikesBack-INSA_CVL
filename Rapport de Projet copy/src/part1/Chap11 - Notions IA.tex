\chapter{Concepts utiles d'Intelligence Artificielle}
    \section{Fonctionnement général du Machine Learning}

    \section{Algorithmes de classification}
        \subsection{Intérêt et fonctionnement de la classification}
        \subsection{Exemples d'algorithme de classification}
            \subsubsection{Régression logistique}
            \subsubsection{Bayésien naïf}
        \subsection{Métriques d'évaluation}
            \subsubsection{Matrices de confusion}
            \subsubsection{ROC et AUC}
            \subsubsection{Validation croisée à $k$-plis}
    \section{Deep Learning}
        Cette section s'appuie essentiellement sur \cite{UDL}.
        \subsection{Définition d'un réseau de neurones}
        \subsection{Hyperparamètres d'un réseau de neurones}
            \subsubsection{Taille du batch}
            \subsubsection{Taux d'apprentissage}
            \subsubsection{Nombre de couches cachées}
            \subsubsection{Taille des couches cachées}
            \subsubsection{Nombre d'échantillons}
        \subsection{Un modèle à deux réseaux : le \textit{Generative Adversarial Network} (GAN)}
            \begin{figure}[H]
                \centering
                \fbox{\begin{tikzpicture}
  [   ->, thick,
    node/.style={circle, fill=teal!60},
    label/.style={below, font=\footnotesize},
  ]

  \node[node] (zin) {$\vec z_\text{in}$};
  \node[node, right=5em of zin] (fake) {$\vec x_\text{fake}$};
  \draw (zin) -- node[above] {$G(\vec x)$} node[label] {generator} (fake);

  \draw[<-] (zin) -- node[above] {$p_\theta(\vec z)$} node[label] {latent noise} ++(-3,0);
  \node[node, above=of fake] (real) {$\vec x_\text{real}$};
  \draw[<-] (real) -- node[above] {$p_\text{data}(\vec x)$} ++(-3,0);
  \node[node, right=6em of fake] (D) at ($(fake)!0.5!(real)$) {$\vec x$};
  \node[right=7em of D] (out) {real?};
  \draw (D) -- node[above] {$D(\vec x)$} node[label] {discriminator} (out);

  \coordinate[right=2.5em of fake, circle, fill, inner sep=0.15em] (pt1);
  \coordinate[right=2.5em of real, circle, fill, inner sep=0.15em] (pt2);

  \draw[-, dashed] (pt1) edge[bend left] coordinate[circle, fill=orange, inner sep=1mm, pos=0.7] (pt3) (pt2);
  \draw (fake) -- (pt1) (real) -- (pt2) (pt3) -- (D);
\end{tikzpicture}}
                \caption{Schéma haut niveau d'un GAN}
            \end{figure}
    \section{Quelques métriques d'évaluation statistique}
        \subsection{Divergence de Kullback-Leibler}
        \subsection{Divergence de Jensen-Shannon}
